\documentclass[dvipdfmx]{jsarticle}

\usepackage[version=3]{mhchem}
\usepackage{amsmath}
\usepackage[siunitx]{circuitikz}
\usepackage{graphicx}
\usepackage{here}

\setlength\parindent{0pt}

\begin{document}
\title{I2実験レポート}
\author{03190449  堀 紡希}
\date{\ 7月11日}
\maketitle

I2実験では主にネットワークについて学習した。

まずifconfigコマンドを用いて自分のIPアドレスを確認した。IPアドレスは192.168.100.107であった。また接続していたwifiをZENKIJIKKENからist-membersに変えるとIPアドレスが変わることも確認した。

次にpingを用いて班員のIPアドレスと疎通確認をした。確認するまでの時間は自分自身にpingを使って送ると相当に短くなった。これはなぜなのかはよくわからなかった。結局ルーターを介して通信先に送るということは同じだと考えたのだが、どうやら違うようだ。(送り先を探す時間の違い?)

次にhostコマンドでいくつかのWebサーバのIPアドレスを確認した。そしてデータの受け渡しをすることができるncコマンドを用いて電気系のページにページのデータを要求し、それを受け取りHTMLで表示することで簡易webクライアントが完成した。

2日目はソケットプログラミングを行なった。
これはネットワークを使った通信をするためのプログラミングインタフェース(API)である。
これを使ってncコマンドで待機しているサーバにデータを送ることができるようになった。

これは基本的に実験テキストに従うことでmanコマンドやインターネットを利用して必要なヘッダを取り込んでC言語で実装することができた。

4日目にはソケットのサーバ側を作成した。

これもクライアント側ができていれば同じような構造なので実装するのにそこまで苦労はなかった。

これに双方向の通信をすることを許すと通話ができるようになった。ただしsoxコマンドのバッファのせいか通信が繋がった時にサーバが待機していた時の音声が流れるという問題が発生した。

そこで通信が繋がった瞬間にsoxコマンドを開始するためにpopenを使ってsoxコマンドを後で呼び出せるようにした。これをplay,recの両方に行うことでより良い通信が行えるようになった。

I3実験ではフーリエ変換をしてバンドパスフィルターをかけて通信料を軽くする、ボイスチェンジャーをつける、といったことに取り組もうと考えている。


\section{参考文献}
[1]東京大学工学部:「電気電子情報第一(前期) 実験テキスト」, 2019.

\end{document}