\documentclass[dvipdfmx]{jsarticle}

\usepackage[version=3]{mhchem}
\usepackage{amsmath}
\usepackage[siunitx]{circuitikz}
\usepackage{graphicx}
\usepackage{here}

\setlength\parindent{0pt}

\begin{document}
\title{20 高電圧実験  考察報告書 }
\author{電子情報工学科 03-190449 堀 紡希}
\date{\ 11月5日}
\maketitle

\section{実験結果}


\section{検討・考察課題}
\subsection*{1.}
相対空気密度、湿度を測定しておく理由は、
気温を$t^\circ$C、気圧を$p$mHg、絶対湿度$h$g/m$^{3}$とすると火花電圧は相対空気密度$\delta = \frac{0.386\times p}{273+t}$、$k = 1 + 0.002\left(\frac{h}{\delta} - 8.5\right)$として、補正後の火花電圧$V_{s}$は$V_{s} = V_{n}\cdot \delta \cdot k$と表されるので、気温、気圧、湿度は電圧の校正に影響を与えるから。

\subsection*{2.}
50Hzの家庭用電圧をそのまま電源とし、試験用変圧器で変換すると、家庭用電圧は、様々なところで電力を消費され、試験用変圧器に歪んで届くので、補正する必要がある。

そこで、試験用変圧器の電源に歪みのない交流を発生させられる、正弦波発電機を使用している。


また、
\subsection*{3.}


\subsection*{4.}


\subsection*{5.}


\subsection*{6.}

\subsection*{7.}

\subsection*{8.}


\section{参考文献}
[1]東京大学工学部:「電気電子情報第一(前期) 実験テキスト」, 2019.

[2]廣瀬明:「電気電子計測」, 数理工学社, 2003.

[3]日高邦彦:「高電圧工学」, 数理工学社, 2013
\end{document}