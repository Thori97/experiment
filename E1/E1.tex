\documentclass[dvipdfmx]{jsarticle}

\usepackage[version=3]{mhchem}
\usepackage{amsmath}
\usepackage[siunitx]{circuitikz}
\usepackage{graphicx}
\usepackage{here}

\setlength\parindent{0pt}

\begin{document}
\title{E1実験考察レポート}
\author{03190449  堀 紡希}
\date{\ 7月2日}
\maketitle

\section{考察課題}

\begin{enumerate}
\item[(1)]
pが大気圧程度である場合の衝突電離係数$\alpha$を表す実験式

実験テキストによるとpが比較的小さい時
\begin{equation}
\frac{\alpha}{p} = A\exp{(-\frac{Bp}{E})}
\end{equation}
で衝突電離係数$\alpha$(1個の電子の単位長あたりの電離回数)が求められる。

一方pが大気圧程度である時
\begin{equation}
\frac{\alpha}{p} = k_{1}(\frac{E}{p}-k_{2})^{2}+k_{3}
\end{equation}
で表される$^{[3]}$。
(2)式では(1)式と違って$\alpha$が$E$の二乗に比例していて(1)よりも増加は急峻である。

\
\item[(4)]窒素分子のエネルギー遷移過程

$\lambda$ = 350[nm], 390[nm]に対応する波数$K$はそれぞれ$K = 8065.5 \times 1239.85/ \lambda$より$K =$ 25641, 28571[$cm^{-1}$]である。

実験テキスト図E1.15でエネルギー準位の中でエネルギー準位同士の差がその範囲にある組を探すと、
SP02(26290cm$^{-1}$),SP13(26637cm$^{-1}$),SP24(26958cm$^{-1}$),SP35(27243cm$^{-1}$)であった。これが実験で観測されたはずである。
これらのエネルギー$K$をジュールに変換して、それを発生させる電子のエネルギーと速度を求める。

\[ E_{d}[eV] = \frac{K[cm^{-1}]}{8065.5}\]
\[ \frac{1}{2}mv^{2} = E_{d}[J]\]
として$m = 9.1\times10^{-31}$[kg], 1eV = $1.6\times10^{-19}$[J]として、おおよそ
$E_{d}= 3.3$[eV], $v = 1.1\times10^{6}$[m/s]必要となる。




\end{enumerate}
\section{参考文献}
[1]東京大学工学部:「電気電子情報第一(前期) 実験テキスト」, 2019.

[2]廣瀬明:「電気電子計測」, 数理工学社, 2003.

[3]日高邦彦:「高電圧工学」, 数理工学社, 2013
\end{document}